%%%%%%%%%%%%%%%%%%%%%%%%%%%%%%%%%%%%%%
\chapter{INTRODUCTION}
%%%%%%%%%%%%%%%%%%%%%%%%%%%%%%%%%%%%%%

%%%%%%%%%%%%%%%%%%%%%%%%%%%%%%%%%%%%%%%%%%%%%%%%%%%%%%%%%%%%%%%%%%%%
\section{Light and Sound}

Lightning bolts piercing the~sky, "magnesian stones" attracting iron shards, echoes reciting actors' words in a~theater --- millennia ago, these and other mysteries were inspiring mankind to seek answers to the~nature of things.
Over the~time three different domains of knowledge~--- electricity, magnetism, and acoustics, --- emerged to encompass the~abovementioned and a~vast number of other similar phenomena.
These domains were regarded as totally independent from one another, describing effects of absolutely different physical nature, and it was not until the~early XIX century when {\O}rsted experimentally discovered the~deflecting effect of an~electric current in a~wire on a~suspended magnetic needle, thus laying the~foundation for a~unified electromagnetic theory.
Further experiments and investigations unveiled the~wave nature of light and established a~connection between the~propagating compressions and rarefactions of media, i.e., the~sound waves, and the~electromagnetic radiation.

Nowadays, the~similarity between the~two only grows, thanks to the~studies of how light and sound interact with complex structures, revealing the~counterparts of electromagnetic effects in acoustics and vice versa.
A good example is the~extraordinary transmission of light \cite{ebbesen} and sound \cite{lu,estrada,christensen1} through subwavelength apertures in a~plate which in both cases is enabled by the~surface waves squeezing the~energy into the~openings.

%%%%%%%%%%%%%%%%%%%%%%%%%%%%%%%%%%%%%%%%%%%%%%%%%%%%%%%%%%%%%%%%%%%%
\section{Interfaces between Media}

The example of extraordinary transmission brings the~question of what role the~boundaries between various parts of a~physical system play.
We all know that boundaries separate regions occupied by different materials, but what does this imply?
The existence of surface waves, which are confined to a~narrow region along the~interface between two media, is the~evidence that the~boundaries are not just auxiliary elements of a~physical system.
In fact, they can be treated as individual objects enabling the~energy transmission in ways that are impossible to realize inside the~bulk.
Moreover, on a~microscopic level the~atoms constituting surface layers of a~material are known to have an~electronic band structure that is completely different from that of the~"bulk" atoms.
Such special states of atoms (Shockley \cite{shockley} or Tamm \cite{tamm} states) cause the~unusual response to the~external excitations, which leads to the~observation of miscellaneous anomalies in wave propagation close to the~surfaces.

%%%%%%%%%%%%%%%%%%%%%%%%%%%%%%%%%%%%%%%%%%%%%%%%%%%%%%%%%%%%%%%%%%%%
\section{Recent Motivations}

The past few decades were plentiful with the~advances in the~areas of nanophotonics and metamaterials.
The nanophotonics research explores ways to concentrate electromagnetic fields of optical frequencies within the~nanometer-sized structures, i.e., within regions shorter than the~wavelength \cite{maier2}, and the~metamaterials provide much needed unconventional material properties to assist with that \cite{shalaev}.
The prospect of accumulating the~energy and thus achieving the~unprecedented field enhancement at nanoscale drives the~development of optoelectronic devices and near-field imaging and conceives new applications for tumor treatment and solar cell design \cite{smallworld,notsosmall,schuller}.

Underlying the~nanophotonics is the~field of plasmonics, which studies the~phenomena associated with surface plasmons --- electric charge excitations coupled with the~electromagnetic field and bound to metallic surfaces.
Interaction of light with surface plasmons makes the~enhancement of linear and nonlinear optical processes possible \cite{stockman}, however, there is a~certain downside to the~plasmon-enabled processes \cite{khurgin1}.
It is known that scaling down the~physical system raises the~questions of surface plasmon radiation \cite{khurgin2,khurgin3} and stability \cite{gumbs}.
The latter dissipative effects obviously limit the~capabilities of the~surface plasmons and ultimately impede the~efficiency of the~plasmonic devices.

In this work, I will direct my attention to a~rather unusual aspect regarding surface plasmons, namely, the~idea of the~surface itself.
The discovery of surface plasmons more than a~century ago \cite{maier2} triggered the~rapid growth of plasmonics, yet it may seem as if the~concept of an~interface between two media froze in time.
In theoretical studies the~interface along which the~surface plasmon propagates is still regarded as an~infinitely thin surface or line separating one medium from another.
Such status quo has to be reconsidered as one descends into the~nanoscale realm, where the~atomic and electronic structures of matter start manifesting themselves.
This brings me to the~textbook problem of the~surface plasmon propagation, which needs to be studied now without the~approximation of an~infinitely thin interface.
Instead, a~more physical interface --- a~relatively thin layer where the~macroscopic material properties change smoothly --- will be incorporated into the~problem, and I will examine its implications on the~properties of the~surface plasmons.

The research problem outlined above does not constitute all of my interest in the~topic as the~diversity of the~effects is beyond imagination.
Moreover, numerous phenomena observed and predicted for plasmonic structures motivate the~exploration of similar systems in acoustics.
With the~speed of sound in any material being orders of magnitude smaller than the~speed of light, the~research returns back to the~macroscopic scale, abolishing the~need to design the~components of the~system with the~nanometer precision.
In particular, the~plasmonic-assisted effects of extraordinary optical transmission through subwavelength apertures \cite{ebbesen} and energy transfer along the~array of nanoparticles \cite{quinten} connect with the~respective effects of anomalous acoustic transmission through a~narrow slit in a~metal plate and through an~array of microperforated shells.
Recent experimental results \cite{adv,garcia1} demonstrate peculiar features in the~transmission spectra, which arise due to excitation of surface localized waves and thus demand a~detailed theoretical analysis.


%%%%%%%%%%%%%%%%%%%%%%%%%%%%%%%%%%%%%%%%%%%%%%%%%%%%%%%%%%%%%%%%%%%%
\section{Dissertation Description}

In this dissertation I study the~phenomena that are caused by the~interaction of sound or light with the~interfaces between media.

The~second chapter presents the~study of the~sound transmission through a~fluid-immersed metallic plate with a~rectangular channel pierced in it.
When the~elasticity of solid is included in the~picture, the~system gains the~ability to support elastic surface waves (Rayleigh waves) at the~metal-fluid interfaces.
In order to calculate the~transmission properties of the~plates with the~channel, I extend the~typical approaches used to address the~diffraction of sound on a~slit and propagation of sound in a~waveguide.
I introduce the~notion of acoustic potentials which are used to derive the~dispersion relation for coupled Rayleigh waves and the~transmission coefficient.
Also, I discuss how different --- propagating and leaky --- coupled Rayleigh modes of the~channel allow propagation and redirection of sound.

In the~third chapter the~process in question is the~scattering of acoustic waves by a~linear chain of weak scattering units, with every unit being a~perforated metal sheet rolled into a~cylindrical shape.
I consider each scatterer in the~approximation of effective impedance, and I formulate the~scattering problem for the~periodic arrangement of perforated shells in cylindrical geometry.
Solving the~scattering problem yields the~transmission spectrum of the~chain and the~dispersion of the~acoustic modes supported by the~chain.
Analysis of both reveals the~correspondence between the~strong suppression of sound transmission and the~excitation of the~chain's eigenmodes.
Based on the~obtained results, I demonstrate how the~system studied can serve as a~passive redirecting antenna or a~passive splitter of sound.

In the~fourth chapter I challenge the~used-by-default model of the~infinitely thin interface between the~two media and look at the~implications of employing the~more realistic model with continuous transitions of media properties.
In particular, I revise the~problem of the~surface plasmon polariton propagation along metal-dielectric interfaces.
I solve Maxwell's equations to obtain the~plasmonic dispersion and electromagnetic field and analytically show that this commonly disregarded transition layer between media actually affects the~properties of the~surface plasmon.
Namely, I establish that the~plasmon within the~transition layer is forced to decay nonradiatively, which adds to the~usual Joule dissipation, and derive the~respective formulas.
I also present numerical simulations of the~surface plasmon excitation by a~laser beam in Kretschmann and Otto configurations to prove that the~spectra of the~reflected light are modified when the~width of the~transition layer is changed.

The~final chapter concludes the~dissertation, summarizes the~work done and explains the~results obtained.


%%% Local Variables: 
%%% mode: latex
%%% TeX-master: "dissertation"
%%% End: 
