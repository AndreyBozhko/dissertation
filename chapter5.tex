%%%%%%%%%%%%%%%%%%%%%%%%%%%%%%%%%%%%%%
\chapter{RESULTS AND CONCLUSIONS}
%%%%%%%%%%%%%%%%%%%%%%%%%%%%%%%%%%%%%%

In this dissertation I investigated the~implications of the~physical processes occurring at the~interfaces between media in three different acoustic and electrodynamic systems.
There are two keystones that allow a~comprehensive study of each system and provide insights to the~nature of phenomena occurring in them.
The first one is the analysis of the~band structure together with the~dispersion relation of the systems, and the other one is the~calculation of the~transmission properties that requires solving the~inhomogeneous problem.

For the~two acoustic systems --- the~fluid channel between two solid plates and the~periodic chain of perforated metallic cylindrical shells --- I analytically obtained their transmission spectra and explained the~reasons for anomalously suppressed transmission at certain frequencies, which was experimentally observed previously.
I also demonstrated how these systems can serve as passive antennas, redirecting or splitting the~incident acoustic signal.
The~latter nontrivial effects are enabled by the~surface modes excited at the~interfaces in the~systems: coupled Rayleigh waves in one case and quasi-surface waves confined along the~chain in another case.

For the~metal-dielectric slab having a~realistic interface (essentially being an~ENZ transition layer) I theoretically derived how the~spectrum of the~surface plasmon propagating along such interface is altered by the~presence of the~transition layer and discovered a~new nonradiative mechanism of plasmonic decay which the~plasmon thus unavoidably acquires.
The~numerical simulations that I performed visualized the~excitation of the~surface plasmon inside the~ENZ layer between the~metal and the~dielectric when the~layer thickness was controlled by an~external electrostatic field.
The~results obtained from the~simulations agreed with the~theoretical analysis and helped understand how the~effect can be observed experimentally.

An~interesting question to discuss is whether the~material properties and system geometries can be optimized in order to achieve stronger manifestation of the~described phenomena.
I conclude that in each case one should run a~corresponding optimization procedure to find the~optimal values instead of simply pushing the~material properties to their extremes.
Namely, the~fluid-channel system is able to redirect sound through the~Rayleigh waves, the~existence of which relies on the~coupling between the~fluid and the~solid plates.
If the~density and stiffness of the~solid are significantly increased, the~plates become rigid and no propagation of Rayleigh waves is possible.
In the~other extreme case, when the~impedance of the~solid $\rho c$ is reduced to as low as the~impedance of fluid, the~whole system becomes virtually transparent for incoming sound, which does not facilitate emerging of Rayleigh waves.
The~same goes for the~linear chain of perforated shells: the~sound is redirected via one of the~eigenmodes which would be much more dissipative if the~shells were thicker or with smaller perforations, or would cease to exist for ultrathin shells with large perforations.
The~splitting of sound by the~chain also relies on the~weakness of the~scatterers, since a~better frequency resolution is achieved when the~band gap is narrower, but, again, the~scatterers too weak would not split any noticeable amount of incoming sound.
In both cases changes to other parameters of the~geometry only lead to scaling of the~device operating frequency regions.
As for the~surface plasmon problem, it is desirable to stretch the~transition layer as much as possible (while not causing a~dielectric breakdown) to get stronger modifications of the~plasmonic spectrum, however, the~accompanying nonradiative losses would grow as well, extinguishing the~plasmon before it is even formed.

Overall, I exposed several physical phenomena which are of fundamental significance to physical acoustics and electrodynamics, although they also have immediate value to bring to the~real-world applications by furnishing design ideas for acoustic waveguides and antennas and establishing the~limits of the~plasmonic-based devices.

%%% Local Variables: 
%%% mode: latex
%%% TeX-master: "dissertation"
%%% End: 
